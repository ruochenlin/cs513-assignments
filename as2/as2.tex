\title{CS 513 Assignment 2}
\author{Ruochen Lin}
\documentclass[11pt]{article}
\usepackage{amsmath,amsfonts,amssymb,amsthm}
\usepackage{mathtools}
\usepackage{commath}
\usepackage{setspace}
% \setmonofont{hack}
\begin{document}
\maketitle
\section{}
\subsection{}
The characteristic equation of matrix $A$ is 
\begin{equation}\begin{split} (1-\lambda)(2-\lambda)+2 = 0,\end{split}\nonumber\end{equation} 
solving which will give us the two eigenvalues of $A$:
\begin{equation}\begin{split} \lambda = \frac{3\pm\sqrt{7}i}2.\end{split}\nonumber\end{equation} 
They both have the same `length' in the complex plane:
$$\abs{\lambda}=\sqrt{(\frac32)^2+(\frac{\sqrt{7}}2)^2}=\sqrt{\frac{16}4}=2,$$
which is the spectral radius of $A$.

\subsection{}
The $l_1$- and $l_{\infty}$-norms of $A$ are easy to find:
\begin{equation}\begin{split}\norm{A}_1=2+1=3,\,\,\norm{A}_\infty=2+2=4.\end{split}\nonumber\end{equation}
Finding the $l_2$-norm, on the other hand, goes hand-in-hand with the SVD of $A$; so we give the result below, but postpone the corresponding calculation to the next subsection:
\begin{equation}\begin{split} \norm{A}_2=2\sqrt{2}.\end{split}\nonumber\end{equation} 

\subsection{}
\begin{equation}\begin{split} 
A^TA&=\begin{bmatrix} 2 & 1\\-2&1 \end{bmatrix}\begin{bmatrix} 2&-2\\1&1\end{bmatrix}=\begin{bmatrix} 5&-3\\-3&5\end{bmatrix},\\
AA^T&=\begin{bmatrix} 2&-2\\1&1\end{bmatrix}\begin{bmatrix} 2 & 1\\-2&1 \end{bmatrix}=\begin{bmatrix} 8 & 0\\0 & 2\end{bmatrix}.
\end{split}\nonumber\end{equation}
Schur decomposing $A^TA$ and $AA^T$ gives us the left and right singular matrices of $A$, $U$ and $V$:
\begin{equation}\begin{split} A^TA&=V\Lambda V^T,\\AA^T&=U\Lambda U^T.\end{split}\nonumber\end{equation} 
Solving the characteristic equation of $A^TA$, 
\begin{equation}\begin{split} (5-\lambda)^2-9=0,\end{split}\nonumber\end{equation}
gives us the eigenvalues of $A^TA$, 
\begin{equation}\begin{split} \lambda_1=2,\,\,\lambda_2=8,\end{split}\nonumber\end{equation} 
and the corresponding normalised eigenvectors:
\begin{equation}\begin{split} v_1 &= \begin{bmatrix} \frac{\sqrt{2}}2 \\ \frac{\sqrt{2}}2\end{bmatrix}, \\
 v_2 &= \begin{bmatrix}-\frac{\sqrt{2}}2 \\ \frac{\sqrt{2}}2\end{bmatrix}.
\end{split}\nonumber\end{equation} 
Thus we have:
\begin{equation}\begin{split} \Lambda&=\begin{bmatrix} 2 & 0\\ 0 & 8\end{bmatrix}, \\
V&=\begin{bmatrix} \frac{\sqrt{2}}2 & -\frac{\sqrt{2}}2 \\ \frac{\sqrt{2}}2 & \frac{\sqrt{2}}2\end{bmatrix}.
\end{split}\nonumber\end{equation} 
The square roots of the eigenvalues gives us the sigular values of $A$: 
\begin{equation}\begin{split} \sigma_1=\sqrt{\lambda_1}=\sqrt{2},\,\,\sigma_2=\sqrt{\lambda_2} = 2\sqrt{2}.\end{split}\nonumber\end{equation}
The largest of the two ($\sigma_2=2 \sqrt{2}$) is also the $l_2$-norm of $A$.\\[0.3cm]
In order to find $U$, we can decompose $AA^T$ in the same fashion; however, since we have figured out $\Lambda$ and $V$, an easier way of finding $U$ would be recognizing that:
\begin{equation}\begin{split} 
Av_1 &= \begin{bmatrix} 2 & -2 \\ 1 & 1\end{bmatrix}\begin{bmatrix} \frac{\sqrt{2}}2 \\ \frac{\sqrt{2}}2\end{bmatrix} = \begin{bmatrix} 0 \\ \sqrt{2}\end{bmatrix}\\
&=\sqrt{2}\,u_1,\\
Av_2 &= \begin{bmatrix} 2 & -2 \\ 1 & 1\end{bmatrix}\begin{bmatrix} \frac{-\sqrt{2}}2 \\ \frac{\sqrt{2}}2\end{bmatrix} = \begin{bmatrix}-2\sqrt{2}\\0\end{bmatrix}\\
&=2\sqrt{2}\,u_2.
\end{split}\nonumber\end{equation} 
Hence 
\begin{equation}\begin{split} U = \begin{bmatrix} 0 & -1 \\ 1 & 0\end{bmatrix} \end{split}\nonumber\end{equation} and
\begin{equation}\begin{split} A = U\Sigma V^T = U\Lambda^{\frac12}V^T\end{split}.\nonumber\end{equation}
Indeed, we find that 
\begin{equation}\begin{split} 
AV &= \begin{bmatrix} 2 & -2\\1 & 1\end{bmatrix}\begin{bmatrix} \frac{\sqrt{2}}2 & -\frac{\sqrt{2}}2 \\ \frac{\sqrt{2}}2 & \frac{\sqrt{2}}2 \end{bmatrix} = 
\begin{bmatrix} 0 & -\sqrt{2} \\ \sqrt{2} & 0\end{bmatrix}\\
&= U\Sigma = \begin{bmatrix} 0 & -1\\1 & 0\end{bmatrix}\begin{bmatrix} \sqrt{2} & 0\\0 & 2 \sqrt{2} \end{bmatrix}   
\end{split}\nonumber\end{equation} 
and that
\begin{equation}\begin{split} 
A^TU &= \begin{bmatrix} 2 &1 \\ -2 & 1\end{bmatrix} \begin{bmatrix} 0 & -1\\1 & 0\end{bmatrix} = \begin{bmatrix} 1 & -2\\ 1 & 2\end{bmatrix} \\
&=V\Sigma = \begin{bmatrix} \frac{\sqrt{2}}2 & -\frac{\sqrt{2}}2 \\ \frac{\sqrt{2}}2 & \frac{\sqrt{2}}2 \end{bmatrix} \begin{bmatrix}\sqrt{2} & 0\\0 & 2\sqrt{2}\end{bmatrix}.
\end{split}\nonumber\end{equation} 

\section{}
My finding this that:


\newpage
\appendix{}
\section{\texttt{MATLAB} code that verifies results in Problem 1}
% TODO: matlab code to verify the results in problem 1
\end{document}
