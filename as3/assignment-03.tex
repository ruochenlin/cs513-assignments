\title{CS 513 Assignment 3}
\author{Ruochen Lin}
\documentclass[11pt]{article}
\usepackage{amsmath,amsfonts,amssymb,amsthm}
\usepackage{mathtools}
\usepackage{commath}
\usepackage{setspace}
% \setmonofont{hack}
\begin{document}
\maketitle
\section{}
Please see attached \texttt{MATLAB} print out.
\section{}
Judging from the graph, the minimum is achieved at $x=0$, and the maximum is achieved somewhere in $(-0.7, 0.5)$ and again at $(0.5, 0.7)$. It is easy to verify that $x=0$ is indeed the global minimum of $f(x)$, because $f(0) = 0$ and $f(x)>0$ when $x\neq0$. To locate the maximum more accurately, we called the \texttt{solve(y1(x)==0, x)} in \texttt{MATLAB}, and the three roots are $-0.5778$, $0$, and $0.5778$. The second root corresponds to the minima, and plugging in $\pm0.5778$ into \texttt{y0} we got the approximate maximum of $f(x)$ on $[-10, 10]$: $f(\pm0.5778) = 0.1228$.
\section{}
\subsection{}
If we evaluate the condition number of $A$ with $l_2$-norm, then 
$$c_2(A) = \sqrt{\frac{\max\{\sigma(A^TA)\}}{\min\{\sigma(A^TA)\}}},$$
because $\sqrt{\max\{\sigma(A^TA) \}}$ and $\sqrt{\min\{\sigma(A^TA) \}}$ gives the largest and smallest sigular values of $A$, respectively, and their ratio is the condition number of $A$.
\subsection{}
If $A$ is symmetric, we have proved in the previous assignment that its sigular values are just the absolute value of its eigenvalues. Thus 
$$c_2(A) = \frac{\max\{\abs{\sigma(A)} \}}{\min\{\abs{\sigma(A)} \}},$$
in which $\abs{\sigma(A)}$ denotes the set of the absolute values of $A$.
\subsection{}
% TODO: check theorem on examples

\section{}
Please see attached \texttt{MATLAB} code and output.
\section{}
\subsection{}
Given that $A = C^TC$, $C\in\mathbb{R}^{m\times m}$, for any $x\in\mathbb{R}^m$ we have 
\begin{equation}\begin{split} 
x^TAx = x^TC^TCx = (Cx)^T(Cx)\geq0,
\end{split}\nonumber\end{equation}
thus $A$ is positive definite. 
\subsection{}
Given $$A = \begin{bmatrix} 2 & 2 & 3 \\ 2 & 4 & 5 \\ 3 & 5 & 8\end{bmatrix}, $$
we can LU-factorize $A$ as the following:
\begin{equation}\begin{split} 
L_1 &= \begin{bmatrix} 1 & 0& 0\\ -1 & 1 &0 \\ -\frac32 & 0 & 1\end{bmatrix},\,\,
L_1A = \begin{bmatrix}2 & 2 & 3 \\ 0 & 2 & 2\\ 0 & 2 & \frac72 \end{bmatrix};\\
L_2 &= \begin{bmatrix} 1 & 0 & 0 \\ 0 & 1 & 0\\ 0 & -1 &1 \end{bmatrix},\,\,
U = L_2L_1A = \begin{bmatrix} 2 & 2 & 3 \\ 0 & 2 & 2 \\ 0 & 0 & \frac32 \end{bmatrix};\\
L &= L_1^{-1}L_2^{-1} = \begin{bmatrix}1 & 0 & 0 \\ 1 & 1 & 0 \\ \frac32 & 1 & 1 \end{bmatrix},\\
A &= LU.
\end{split}\nonumber\end{equation} 
If we further factorize $U$ into the product of a diagonal matrix $D$ and a unit upper triangular matrix $\tilde U$, we have:
\begin{equation}\begin{split} 
D &= \begin{bmatrix} 2 & 0 & 0 \\ 0 & 2 & 0 \\ 0 & 0 & \frac32\end{bmatrix}, \,\, 
\tilde U = \begin{bmatrix} 1 & 1 & \frac32\\ 0 & 1 & 1 \\ 0 & 0 & 1\end{bmatrix},\\
A &= L D \tilde U.  
\end{split}\nonumber\end{equation} 
\subsection{}
We notice that $\tilde U = L^T$, so if we write $D$ as the square of a diagonal matrix $\tilde D$, then $C = \tilde D \tilde U$. There are actually $2^3=8$ possible choices of $\tilde D$, since each of its three diagonal entries can carry either $+$ or $-$ sign. The following is one of the viable choices:
\begin{equation}\begin{split}
\tilde D &= \begin{bmatrix} \sqrt{2} & 0 & 0 \\ 0 & \sqrt{2} & 0 \\ 0 & 0 & \frac{\sqrt{6}}2 \end{bmatrix},\\
C &= \tilde D \tilde U = \begin{bmatrix}\sqrt{2} & \sqrt{2} & \frac{3\sqrt{2}}2 \\ 0 & \sqrt{2} & \sqrt{2} \\ 0 & 0 & \frac{\sqrt{6}}2  \end{bmatrix},\\
A &= C^TC.
\end{split}\nonumber\end{equation} 
And this $C$ is also the choice of \texttt{MATLAB}.
% TODO: add chol matlab screenshot
\section{}
Please see attached code and output for solutions.
% TODO: discuss whether it's a contradiction.
\end{document}
