\title{CS 513 Assignment 4 -- Part II}
\author{Ruochen Lin}
\documentclass[11pt]{article}
\usepackage{amsmath,amsfonts,amssymb,amsthm}
\usepackage{mathtools}
\usepackage{commath}
\usepackage{setspace}
\usepackage{algorithm,algorithmic}
% \setmonofont{hack}
\begin{document}
\maketitle
\section{}
\subsection{}
\begin{equation}\begin{split} 
A = \begin{bmatrix} 
309 & 228 & -240 \\
60 & -117 & 510 \\ 
12 & 6 & 298
\end{bmatrix} &/49,\,\,v = \begin{bmatrix} 6 \\ 3 \\ 2\end{bmatrix} \\
Av = \begin{bmatrix} 42 \\ 21 \\ 14 \end{bmatrix} &= 7v,
\end{split}\nonumber\end{equation}
thus $(7,v)$ is an eigenpair of $A$.\\[0,35cm]
First we can normalize $v$ and define 
$$v_0 = \frac v{\norm{v}_2} = \begin{bmatrix} 6/7 \\ 3/7 \\2/7\end{bmatrix}. $$
The Householde matrix that maps $\tilde v$ to $e_1$ can be constructed as the following:
\begin{equation}\begin{split} 
H &= I - \frac2{\norm{v_0 - e_1}^2}(v_0 - e_1)(v_0 - e_1)^T\\
&=\begin{bmatrix} 
\frac67 & \frac 37 & \frac27 \\
\frac37 & -\frac 27 & -\frac67 \\
\frac27 & -\frac67 & \frac37
\end{bmatrix},\\
B&=HAH=\begin{bmatrix} 
7 & 0 & 0 \\
0 & 9 & -6 \\
0 & 6 & -6
\end{bmatrix}.
\end{split}\nonumber\end{equation}
We see that 7 is indeed the first entry of $B$, just as expected. Also, the deflated matrix is
$$B_1 = \begin{bmatrix}9 & -6\\6 & -6 \end{bmatrix}.$$

\subsection{}
\begin{equation}\begin{split} 
\det(B_1-\lambda I) &= (9-\lambda)(-6-\lambda) + 36 = 0 \\
\Longrightarrow \lambda_1 &= -3,\,\,\lambda_2 = 6.
\end{split}\nonumber\end{equation} 
Solving $(B_1-\lambda_iI)w_i=0$, requiring $w_i$ to be normalized, we have:
\begin{equation}\begin{split} 
w_1 = \begin{bmatrix} \frac1{\sqrt{5}} \\ \frac2{\sqrt{5}}\end{bmatrix},\,\,
w_2 = \begin{bmatrix} \frac2{\sqrt{5}} \\ \frac1{\sqrt{5}}\end{bmatrix}. 
\end{split}\nonumber\end{equation} 

\subsection{}
Since the row vector right to 7 in the first row of $B$ is zero, the eigenvectors of $B$, corresponding to $\lambda_1$ and $\lambda_2$ are simply
\begin{equation}\begin{split} 
\tilde w_1 = \begin{bmatrix}0 \\ \frac1{\sqrt{5}} \\ \frac2{\sqrt{5}} \end{bmatrix},\,\,\tilde w_2 = \begin{bmatrix} 0 \\ \frac2{\sqrt{5}} \\ \frac1{\sqrt{5}} \end{bmatrix}.
\end{split}\nonumber\end{equation} 
The eigenbasis of $A$, ${v_0,v_1, v_2}$, can be found using
$$v_1 = H\tilde w_1 = \begin{bmatrix} 1 \\ -2 \\ 0\end{bmatrix},\,\,v_2=H\tilde w_2 = \begin{bmatrix} \frac87\\ -\frac{10}7\\ -\frac97\end{bmatrix}.$$ 

\subsection{}
If we concatenate the three eigenvectors of $A$, we have
$$P = \begin{bmatrix}6 & 1 & 8\\ 3 & -2 & -10 \\ 2 & 0 & -9\end{bmatrix},$$
inverting which would gets us 
$$P^{-1} = \begin{bmatrix} 
\frac6{49} & \frac3{49} & \frac2{49} \\
\frac{1}{21} & -\frac{10}{21} & -\frac{4}{7} \\
\frac4{147} & -\frac2{147} & -\frac5{49}
\end{bmatrix}. $$
These, together with
$$
D = \begin{bmatrix} 7 \\ & -3 \\ & & 6 \end{bmatrix} 
$$
give us $A = P D P^{-1}$
\end{document}
