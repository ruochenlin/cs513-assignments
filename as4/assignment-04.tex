\title{CS 513 Assignment 4}
\author{Ruochen Lin}
\documentclass[11pt]{article}
\usepackage{amsmath,amsfonts,amssymb,amsthm}
\usepackage{mathtools}
\usepackage{commath}
\usepackage{setspace}
\usepackage{algorithm,algorithmic}
% \setmonofont{hack}
\begin{document}
\maketitle
\section{}
\subsection{}
In the $k+1$th step of Gaussian elimination for a $m\times m$ square matrix, the first $k$ columns and rows are untouched; thus we are effectively performing the step on a $(n-k)\times(n-k)$ square matrix. If we call this square matrix $B$, and the resulting $(n-k-1)\times(n-k-1)$ matrix $C$, then the entries of $C$ is given by the following equation:
\begin{equation}\begin{split} 
C_{i,j}&=B_{i+1,j+1}-\frac{B_{i+1,1}}{B_{1,1}}B_{1,j+1}\\
\Leftrightarrow C_{j,i} &= B_{j+1,i+1} - \frac{B_{j+1,1}}{B_{1,1}}B_{1,i+1}.
\end{split}\nonumber\end{equation} 
If $B$ is symmetric, then we have $B_{i+1,j+1} = B_{j+1,i+1}$, $B_{i+1,1} = B_{1,i+1}$, and $B_{1,j+1} = B_{j+1,1}$, which leads to $C_{i,j} = C_{j,i}$. In other words, if we start with a symmetric matrix at the beginning of the step, then the resulting matrix will also be symmetric. Note that we do start with a symmetric matrix, $A$; thus the lower right sqaure matrices after each step of Gaussian elimination will all be symmetric.

\subsection{}
We can modify the algorithm of the LU-factorization for symmetric matrices, utilizing the theorem in the preceeding part: in the $k$th step of the Gaussian elimination, we can only update the entries in the lower triangular part of the remaining $(m-k-1)\times(m-k-1)$ block at bottom right. The pseudocode of such algorithm is the following:
\begin{algorithm}[H]
\caption{LU-factorization for a symmetric matrix A}
\begin{algorithmic}
	\STATE L = eye(m)
    \FOR{i = 1 : m - 1}
		\FOR{j = i + 1 : m}
			\STATE L(j,i) = A(j,i) / A(i,i)
			\STATE A(j,i) = 0
			\FOR {k = i + 1 : j}
				\STATE A(j,k) = A(j,k) - L(j,i) * A(i,k)
				\IF{k != j}
					\STATE A(k,j) = A(j,k)
				\ENDIF
			\ENDFOR
		\ENDFOR
    \ENDFOR
	\STATE U = A
\end{algorithmic}
\end{algorithm}
It's easy to see that the most executed part of the algorithm is the inner most loop with the dummy index $k$. In lieu of the analysis in class, if we only count multiplications, then each execution of the inner most loop contributes exactly $1$ operation. The total number of multiplications, $N(m)$, can be estimated as the following:
\begin{equation}\begin{split} 
N(m) &= \sum_{i=1}^{m-1} \sum_{j=i+1}^m \sum_{k=i+1}^j 1\\
&= \sum_{i=1}^{m-1} \sum_{j=i+1}^m (j-i) \\
&= \sum_{i=1}^{m-1} \Big[ 1 + 2 + \cdots + (m-i) \Big] \\
&= \sum_{i=1}^{m-1} \frac{(m-i)(m-i+1)}2 \\
&= \sum_{i=1}^{m-1} \Big[ \frac{(m-i)^2}2 + O(m) \Big]\\
&= \frac12\Big[ 1^2 + 2^2 + \cdots + (m-1)^2 \Big] + O(m^2) \\
&= \frac12\cdot \frac{(m-1)m(2m-1)}6 + O(m^2)\\
&=\frac{m^3}6 + O(m^2).
\end{split}\nonumber\end{equation} 
Thus, our current algorithm is almost twice as fast as Gaussian eliminationi for general matrices.
\end{document}
