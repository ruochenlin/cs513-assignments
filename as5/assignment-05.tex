\title{CS 513 Assignment 5}
\author{Ruochen Lin}
\documentclass[11pt]{article}
\usepackage{amsmath,amsfonts,amssymb,amsthm}
\usepackage{mathtools}
\usepackage{commath}
\usepackage{setspace}
% \usepackage{algorithm,algorithmic}
\begin{document}
\maketitle
\section{}

\section{}
\subsection{}
The cardinality of $\sigma(A)$ is n, because in each of the pairwise disjoint disks there is one (and only one) eigenvalue of $A$, which add up to give $n$ distinct eigenvalues. In addition, for each of the eigenvalues, both the corresponding algebraic and geometric multiplicity is $1$.

\subsection{}
The proposition is true, because for each of the eigenvalues of $A$, the algebraic multiplicity is equal to geometric multiplicity, which is a sufficient (and necessary) condition for $A$ to be diagonalizable.

\subsection{}
The claim is write, and we can prove if by mathematical induction. Starting with a general real $2\times2$ matrix
\begin{equation}\begin{split}
A_2 &= \begin{bmatrix} a & b \\ c & d \end{bmatrix},\\ 
\det(\lambda I - A_2) &= (\lambda-a)(\lambda - d) - bc \\
&= \lambda^2 - (a + d)\lambda + ad - bc. 
\end{split}\nonumber\end{equation} 
For the characteristic equation to have complex solutions, necessarily its discriminant needs to be negative. That is, we require
\begin{equation}\begin{split}
\Delta &= (a+d)^2 -4(ad-bc) \\
&= (a-d)^2+4bc < 0.
\end{split}\nonumber\end{equation} 
However, for the Gershgorin disks to be disjoint, we have 
\begin{equation}\begin{split} 
\abs{a-d} &> \abs{b} + \abs{c}\\
\Rightarrow \,\, (a-d)^2 &> (\abs{b}+\abs{c})^2 \\
&=b^2+c^2+2\abs{bc} \\
&\geq b^2+ c^2 - 2bc.
\end{split}\nonumber\end{equation} 
This implies that 
\begin{equation}\begin{split} 
\Delta& = (a-d)^2+4bc \\
&>b^2+c^2-2bc+4bc \\
&=(b-c)^2 \geq0.
\end{split}\nonumber\end{equation} 
Hence for $A_2\in\mathbb{R}^{2\times2}$, all eigenvalues must be real. Let's move on to $A_3\in\mathbb{R}^{3\times3}$: for $A_3$ to satisfy our requirements, \textit{i.e.} all of the Gershgorin disks of $A_3$ to be mutally disjoint, necessarily its top left $2\times2$ block also satisfies the condition and thus have real eigenvalues and trace. The trace of $A_3$ is the sum of the trace of its upper left $2\times2$ block and $(A_3)_{33}$. Since 
\end{document}
